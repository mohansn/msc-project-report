\section {Nanomaterials and Nanostructures}
A nanostructure is defined as a material system or object, where
at least one of the dimensions lies below 100 nm. Nanostructures
can be classified into three different categories:
\begin{enumerate}
\item zero-dimensional (0D)
\item one-dimensional (1D)
\item two-dimensional (2D)
\end{enumerate}
0D nanostructures are materials in which all three dimensions are at the nanoscale. A good example of these materials are buckminster fullerenes  and quantum dots. 1D nanostructures are materials that have two physical dimensions in the nanometer range, while the third dimension can be large, such as in the carbon nanotube. 2D nanostructures, or thin films, only have one dimension in the nanometer range and are used readily in the processing of complimentary metal-oxide semiconductor transistors and micro-electro-mechanical systems (MEMS). Nanomaterials are the base material of many nanoscale objects. Recently various one-dimensional nanostructures have been realized. They include nanodots, nanorods, nanowires, nanobelts, nanotubes, nanobridges and nanonails, nanowalls, nanohelices, seamless nanorings. Among all the one-dimensional nanostructures, nanotubes, nanorods and nanowires are widely studied. This is because of the easy material formation and device applications.

Nanostructures have unique properties when compared to their individual atoms or
molecules or their bulk macroscopic properties. For example, bulk material such as
Copper wire, their intrinsic properties, say density or conductivity, are independent
of its size. That is, if a 1 m long Copper wire, when cut into few pieces, and for these pieces, if the density or conductivity is measured, one will find they are same as the original Copper wire. If the dividing process is done indefinitely, then the property invariance will still remain. However, if the division is made at the electron, proton or neutron levels, that is at the nanoscale levels, we can certainly expect significant change in the property of the nanostructures. The properties of the nanoscale material systems can get significantly affected by the following three phenomenon:
\begin{itemize}
\item \textit{Quantum Confinement}: The confinement of electrons in the nanoscale dimensions will result in the change in the energy and momentum of the nano material system, which in turn significantly alters its properties.
\item \textit{Quantum Coherence}: : This phenomenon relates to the phase relation of the wave function in nano material system, that is preserved in the nanomaterial system. The quantum coherence property is well maintained in atoms and molecules but not always in nanostructures due to inherent defects present in these structures.
This results in the change of properties in the nano scale and hence it is necessary
to consider both the quantum coherence and de-coherence effects while dealing
with nanostructures.
\item \textit{Surface Effects}: Vast majority of the atoms in a nanostructures are located either at the surfaces or interfaces. The properties of these surface atoms can be quite different than that of those, which are located in the interior.
\end{itemize}

The above factors significantly alter the properties of the nanostructures as compared to their bulk material. For a nanomaterial systems, both the crystalline state and surface/interface state is very important. These materials are often in metastable state. Their atomic configuration depends on the kinetic process in which they are fabricated or grown. Therefore, the properties of nanostructures can be adjusted or manipulated by changing its size, shape or the process by which it is made, which can often lead to some rich and surprising outcomes.

\section {Carbon Nanotubes}
Carbon is a remarkable element that has a unique structures which make it amenable
to combine with other elements and compounds to get a new compound. It is said
that carbon has the ability for form close to 10 million different compounds. It is
present in the food we eat, the clothes we wear, the cosmetics we use and also in
the fuel that drives our cars. Carbon exists in four different allotropes, namely the
amorphous, the graphite, the diamond and the fullerene. The Amorphous carbon
structure is visually a highly disordered structure. It is for this reason that it lacks
structural integrity . This carbon structure forms at the edges or is the residue of other
elemental compounds. The disorder of this structure allows it to have many available
bonds and is responsible for building more complex carbon based molecules.

There are many definitions to nanotubes. The simplest definition of nanotube is that
it is a nanometer scale structure that resembles a tube. There are both organic and
inorganic nanotubes. Organic nanotubes are the carbon nanotubes or CNT. With
respect to Carbon NanoTube (CNT), a nanotube can be defined as \emph{a long cylindrical carbon structure
consisting of hexagonal graphite molecules attached at the edges}. Some nanotubes
have a single cylinder while others have two or more concentric cylinders. Nanotubes
have several characteristics, namely wall thickness, number of concentric cylinders,
cylinder radius, and cylinder length. Some nanotubes have a property called chirality,
an expression of longitudinal twisting.

\subsection {Types and structures}
\subsection {Why study Carbon nanotubes?}
\subsection {Why study wave propagation in CNTs?}
Increasing emphasis of miniature devices have made the scientists to look for newer
and novel materials which can be handled at the atomistic scales. In this regard,
Nanoscale materials and structures with nano thicknesses have attracted consider-
able interest from the scientific community in the fields of microelectronics and
nanotechnology. More and more nanostructures, e.g. ultra-thin films, nanowires
and nanotubes, have been fabricated and served as the basic building blocks for
nano-electro-mechanical-systems (NEMS). For long-term stability and reliability of
various devices at nanoscale, researchers should possess a deep understanding and
knowledge of mechanical properties of nano-materials and -structures, especially the
time dependent or dynamic properties.

Nanostructures such as CNTs can propagate waves of the order of terahertz (THz). As dimensions of the ma-
terial become smaller, however, their resistance to deformation is increasingly deter-
mined by internal or external discontinuities (such as surfaces, grain boundary, strain
gradient, and dislocation). Although many sophisticated approaches for predicting
the mechanical properties of nanomaterials have been reported, few addressed the
challenges posed by interior nanostructures such as the surfaces, interfaces, struc-
tural discontinuities and deformation gradient of the nanomaterials under extreme
loading conditions. The use of atomistic simulation may be a potential solution in
the long run. However, it is well known that the capability of this approach is much
limited by its need of prohibitive computing time and an astronomical amount of data
generated in the calculations. Wave propagation analysis using continuum models,
especially using non-local elasticity models can used to address the above problems.

Wave propagation studies mainly include the estimation of wavenumber and wave
speeds such as phase and group speeds. The concept of group velocity may be useful
in understanding the dynamics of carbon nanotubes, since it is related to the energy
transportation of wave propagation. It is useful to study the
wave propagation in nanostructures, so as to examine the effect of length scales on
the wave dispersion from the viewpoint of group velocity or energy transportation. To
describe the effect of microstructures of a nanostructures on its mechanical properties,
it is assumed that the model of the nanostructure is made of a kind of non-local elastic
material, where the stress state at a given reference location depends not only on the
strain of this location but also on the higher order gradient of strain, so as to take
the influence of the microstructures into account. It is reported that both local elastic
models (where effects of nano scale is not considered) and non-local elastic models
(where the effect of scale is considered) can offer the correct prediction when the
wavenumber is lower. However, the results of the elastic model remarkably deviate
from those given by the non-local elastic model with an increase in the wavenumber.
As a result, the microstructures play an important role in the dispersion of waves in
nanoscale structures. Since terahertz physics of nanoscale materials and devices are
the main concerns in wave characteristics of CNTs, the small-scale effect must be
of significance in achieving accurate dispersion relations as the wavelength in the
frequency domain is in the order of nanometers.


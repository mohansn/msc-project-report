\section {Nanomaterials and Nanostructures}
A nanostructure is defined as a material system or object, where
at least one of the dimensions lies below 100 nm. Nanostructures
can be classified into three different categories: zero-dimensional
(0D); one-dimensional (1D); two-dimensional (2D). 0D nanostructures are materials in which all three dimensions are at the nanoscale. A good example of these materials are buckminster fullerenes  and quantum dots. 1D nanostructures are materials that have two physical dimensions in the nanometer range while the third dimension can be large, such as in the carbon nanotube. 2D nanostructures, or thin films, only have one dimension
in the nanometer range and are used readily in the processing of complimentary metal-oxide semiconductor transistors and micro-electro-mechanical systems (MEMS). Nanomaterials are the base material of many nanoscale objects. Recently various one-dimensional nanostructures have been realized. They include nanodots, nanorods, nanowires, nanobelts, nanotubes, nanobridges and nanonails, nanowalls, nanohelices, seamless nanorings. Among all the one-dimensional nanostructures, nanotubes, nanorods and nanowires are widely studied. This is because of the easy material
formation and device applications.


\section {Carbon Nanotubes}
\subsection {Types and structures}
\subsection {Why study Carbon nanotubes?}
\subsection {Why study wave propagation in CNTs?}

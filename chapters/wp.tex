\section {Fundamental Concepts}
Wave propagation is any of the ways in which waves travel.
With respect to the direction of the oscillation relative to the propagation direction, we can distinguish between longitudinal wave and transverse waves.
For electromagnetic waves, propagation may occur in a vacuum as well as in a material medium. Other wave types cannot propagate through a vacuum and need a transmission medium to exist.\cite{wiki:wp}

\subsection*{Wavenumber and Wave Frequency \cite{mitocwwp}}
If the range of the spatial corrdinate x is $(-\infty,\infty)$,and all coefficients are independent of $x,t$,  then the first task is to examine the physics
of sinusoidal wave train of the form:\\
\[V(x,t) = |A|cos(kx - \omega t - \phi_A)=\mathfrak{R}(A e^{ikx-i\omega t})\]
where $A = |A| e^{i \phi_A}$ is a complex number with magnitude $|A|\text{ and phase angle} \phi_A$ After
examining the physical meaning of this special type of waves, it is possible to use the
principle of superposition to construct more general solutions. It is customary to omit
the symbol $\mathfrak{R} = $ ``the real part of''  for the sake of brevity, i.e.,\\
\[V(x,t) = Ae^{ikx-i\omega t}\]
\emph{Definition}: We shall call:\\
\begin{center}
\framebox{$\theta(x,t) = kx - \omega t$}\\
\end{center}
the \emph{wave phase}. Clearly the trigonometric function is periodic in phase with the period $2 \pi$. In the $x,t$ plane, $V$ has a constant value along a line of contant phase. In particular, $\theta = 2n \pi$ correspond to the wave crests where $V = |A|$ is the greatest.
On the other hand, $\theta = (2n+1) \pi (n=1,2,3,\ldots$ correspond to the wave troughs where $V = -|A|$ is the smallest. $|A|$ is half of the separation between adjacent crests
and troughs and is called the \emph{wave amplitude}; we also call A the complex amplitude. Clearly $\frac{\partial \theta}{\partial x}$ represents the number of phase lines per unit distance, i.e., the density of
phase lines, at a given instant; it is called the \emph{wave number}.\\
\begin{center}
\framebox{$\text{wave number} = k = \frac{\partial \theta}{\partial x}$}
\end{center}
On the other hand $-\frac{\partial \theta}{\partial t}$ represensts the number of phase lines passing across a fixed x per unit time; it is called the \emph{wave frequency}.\\
\begin{center}
\framebox{$\text{wave frequency} = \omega = \frac{\partial \theta}{\partial t}$}
\end{center}
To stay with a particular line of contant phase, say a crest, one must have\\
\[d\theta = k dx -\omega dt = 0\]
namely one must move at the \emph{phase velocity},\\
\begin{center}
\framebox{$c = \left.\frac{dx}{dt}\right|_{\theta=constant} = \frac{\omega}{k}$}
\end{center}
In general a sinusoidal wave whose
phase velocity depends on the wavelength, i.e.,$\omega$ is a nonlinear function of $k$, is called a \emph{dispersive wave}. We define the \emph{group velocity} as:
\begin{center}
\framebox{$\text{group velocity} = c_g = \frac{d\omega}{dt}$}
\end{center}

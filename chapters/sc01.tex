\section{Terahertz Wave Propagation in uniform nanorods}
\subsection*{Introduction}
The dynamic testing of materials and components often involves predicting the propagation of stress waves in slender rods. The present work deals with the analysis of the wave propagation characteristics of nanorods. The nonlocal elasticity theory and also the lateral inertia are incorporated into classical/local rod model to capture unique features of the nanorods under the umbrella of continuum mechanics theory.
The strong effect of the nonlocal scale has been obtained which leads to substantially different wave behaviors of nanorods from those of macroscopic rods. Nonlocal rod/bar model is developed for nanorods including the lateral inertia effects. The analysis shows that the wave characteristics are highly overestimated by the classical rod model, which ignores the effect of small-length scale. The wave propagation properties of the nanorod obtained from the present formulations are compared with the continuum rod model, nonlocal second and fourth order strain gradient models, Born-Karman model and the nonlocal
stress gradient model. It has also been shown that, the unstable second order strain gradient model can be replaced by considering the inertia gradient terms in the formulations. The effects of both the nonlocal scale and the diameter of the nanorod on spectrum curves are studied.

\subsection{Nonlocal governing partial differential equation for nanorods}
Figure \ref{nanorod} schematically describes a nanorod under discussion and
serves to introduce the axial coordinate x, lateral coordinate y, the
axial displacement $u = u(x,t)$the Young’s modulus $E$, the density $\rho$ the Poisson’s ratio $\nu$ and cross-sectional area $A$ . The displacement
field (X-direction), strain, strain rate and particle velocity associated with the displacement field in X-direction for this nanorod
are given by\\
\begin{equation}
u=u(x,t)
\end{equation}
\begin{equation}
\varepsilon_{xx}=\frac{\partial u}{\partial x}
\end{equation}
\begin{equation}
\dot{\varepsilon_{xx}}=\frac{\partial u}{\partial t} = \frac{\partial^2 u}{\partial x \partial t}
\end{equation}
\begin{equation}
V = \dot{u} = \frac{\partial u}{\partial t}
\end{equation}
\begin{figure}[b]
\centering
\includegraphics[scale=0.3]{nanorod}
\caption{A nanorod, showing Young’s modulus $E$, density $\rho$, Poisson’s ratio $\nu$, diameter $d$, cross-sectional area $A$, longitudinal displacement $u=u(x,t)$, strain along X-direction $\varepsilon_x$ and strains along Y-direction $\varepsilon_y$}
\label{nanorod}
\end{figure}
Due to Poisson’s ratio $\nu$, there are displacement fields $\nu$ and $w$ in
Y- and Z-directions, respectively. For example the strain and the
derivative of the displacement with time in the Y-direction are,
respectively,\\
\begin{equation}
\varepsilon_{yy}=-\nu \varepsilon_{xx}
\end{equation}
\begin{equation}
\dot{\nu}=-\nu y\dot{\varepsilon_{xx}}
\end{equation}
The kinetic energy of the infinitesimal length $\delta$ of the rod (from \ref{nanorod}) is:\\
\begin{equation}
\Delta \Pi^e = \frac{1}{2} \rho A \Delta x (V^2+\nu^2 \xi^2 \dot{\varepsilon}^2_{xx})
\end{equation}
where $\rho$ is the density, $A$ is the cross-sectional area and $\xi$ is the
radius of gyration of the solid circular cross-section. Here, an
effective density is used to incorporate the effect of lateral inertia in
a one-dimensional nonlocal wave equation. An effective density $\rho_{eff}$ is now introduced such that the kinetic energy of the element $\Delta x$ is $\Delta \Pi^e_{eff}$, where\\
\begin{equation}
\Delta \Pi^e_{eff} =\frac{1}{2} \rho_{eff} A \Delta x V^2
\end{equation}
Using Newton’s second law, the net longitudinal force acting on the
element $\Delta x$ (see \ref{nanorod}) is:\\
\begin{equation}
\Delta \sigma_{xx} \times A = \rho_{eff} A \frac{\partial^2 u}{\partial t^2} \times x
\label{rhoeff}
\end{equation}
In the limiting case as $\Delta x \rightarrow 0$, equation \ref{rhoeff} becomes:\\
\begin{equation}
\frac{\partial \sigma_{xx}}{\partial x} = \rho_{eff} \frac{\partial^2 u}{\partial t^2}
\label{limcasesigma}
\end{equation}
To find the relationship between $\rho_{eff}$ and $\rho$, a functional $\cup_e$ is defined as the kinetic energy error when using the effective density, i.e.,\\
\begin{equation}
\bigcup_e = \iint [\Delta \Pi^e - \Delta \Pi^e_{eff} ] dx dt = 
\iint \left[ \dfrac{1}{2} \rho A (V^2 + \nu^2 \xi^2 \dot{\varepsilon}^2_{xx}) - 
\left( \dfrac{1}{2} \rho_{eff} A V^2 \right) \right] dx dt
\label{functionaldef}
\end{equation}
substituting for strain rate and particle velocity in \eqref{functionaldef} the
minimization of $\cup_e$ requires minimization of the following functional:\\
\begin{equation}
\bigcap_e = \iint \left\lbrace (\rho - \rho_{eff}) \left[\dfrac{\partial u}{\partial t}\right]^2 + \rho \nu^2 \xi^2 \left[ \dfrac{\partial^2 u}{\partial x \partial t} \right]^2 \right\rbrace dx dt
\end{equation}
The integrand of the functional $\cap_e$ is\\
\begin{equation}
f\left(\dfrac{\partial u}{\partial t},\dfrac{\partial^2 u}{\partial x \partial t}\right) = (\rho - \rho_{eff}) \left[\dfrac{\partial u}{\partial t}\right]^2 + \rho \nu^2 \xi^2 \left[ \dfrac{\partial^2 u}{\partial x \partial t}\right]^2
\label{func2}
\end{equation}
In order to minimize $\cap_e$ in \eqref{func2}, the integrand must satisfy the following equation:\\
\begin{equation}
\dfrac{\partial^2}{\partial x \partial t} 
\left[
 \dfrac{\partial f \left(\frac{\partial u}{\partial t},\frac{\partial^2 u}{\partial x \partial t}  \right)}{\partial \frac{\partial^2 u}{\partial x \partial t}}  
\right] 
- 
\dfrac{\partial}{\partial t}
\left[ 
\dfrac{\partial f \left(\frac{\partial u}{\partial t},\frac{\partial^2 u}{\partial x \partial t}  \right)}{\partial \frac{\partial u}{\partial t}}  
\right] = 0
\label{func3}
\end{equation}
Therefore, \eqref{func3} is identical to the following partial differential equation:\\
\begin{equation}
\rho_{eff} \frac{\partial^2 u}{\partial t^2} = \rho \frac{\partial^2 u}{\partial t^2} - \rho \nu^2 \xi^2 \frac{\partial^4 u}{\partial x^2 \partial t^2}
\label{effectivedenseq}
\end{equation}
The equation \eqref{effectivedenseq} gives the relationship between the effective density
S
and the density that minimizes the effective density error,$\cup_e$ defined in \eqref{functionaldef},over the length of the nanorod and the time of
motion. Substituting this expression in Eq. \eqref{limcasesigma} gives\\
\begin{equation}
\dfrac{\partial \sigma_{xx}}{\sigma x} = \rho \dfrac{\partial^2 u}{\partial t^2} - \rho \nu^2 \xi^2 \dfrac{\partial^4 u}{\partial x^2 \partial t^2}
\end{equation}
The constitutive model employed here is that obtained from the
theory of nonlocal/nonclassical continuum mechanics. For thin
rods Eq. \eqref{NLconsteqn} can be written in the following one dimensional form\\
\begin{equation}
\sigma_{xx} = \alpha^2 \frac{\partial^2 \sigma_{xx}}{\partial x^2} = E \varepsilon_{xx} = E \frac{\partial u}{\partial x}
\label{onedimconst}
\end{equation}
where $E$ is the modulus of elasticity,$\sigma_{xx}$ and $\varepsilon_{xx}$ are the local stress
and strain components in the $x$ direction, respectively, and $\alpha=e_0 a$ nonlocal scaling parameter. Differentiating the Eq. \eqref{onedimconst}with respect to $x$ on both sides, gives\\
\begin{equation}
\frac{\sigma_{xx}}{\partial x} - \alpha^2 \frac{\partial^3 \sigma_{xx}}{\partial x^3} = E \frac{\sigma_{xx}}{\partial x} = E \frac{\partial^2 u}{\partial x^2}
\label{diffonedimconst}
\end{equation}
Substituting equation \eqref{effectivedenseq} in eq. \eqref{diffonedimconst} gives\\
\begin{equation}
E \dfrac{\partial^2 u}{\partial x^2} = \rho \dfrac{\partial^2 u}{\partial t^2} - \rho \nu^2 \xi^2 \dfrac{\partial^4 u}{\partial x^2 \partial t^2} - \alpha^2 \rho \dfrac{\partial^4 u}{\partial x^2 \partial t^2} + \alpha^2 \rho \nu^2 \xi^2 \dfrac{\partial^6 u}{\partial x^4 \partial t^2}
\label{nonlocalinertia}
\end{equation}
Equation \eqref{nonlocalinertia} is the consistent fundamental governing equation of
motion for nonlocal rod model including the effect of lateral inertia/
Poisson’s effect. When $\alpha = e_0 a = 0\text{ and } \nu = 0$ , it is reduced to the equation of local or classical rod model.

\subsection*{Terahertz wave characteristics of nanorods}
For analyzing the ultrasonic/terahertz wave dispersion char-
acteristics in nanorods, we assume that a harmonic type of wave
solution for the displacement field u(x,t) and it can be expressed in
complex form as \cite{gopalakrishnan2008spectral,doyle1989wave}\\
\begin{equation}
u(x,t) = \sum_{p=0}^{P-1} \sum_{q=0}^{Q-1}\hat{u}(x,\omega_q)e^{-(k_p x - \omega_q t)}
\label{harmonicsoln}
\end{equation}
where $P$ and $Q$ are the number of time sampling points and number
of spatial sampling points, respectively, and $j=\sqrt{-1}$. $\omega_q$ is the
circular frequency at the $q$th time sample. Similarly, $k_p$ is the axial
wavenumber at the $p$th spatial sample point. Substituting Eq. \eqref{harmonicsoln} into the governing partial differential equation (eq. \eqref{nonlocalinertia}) we get
the characteristic equation (dispersion relation). The dispersion
relation is solved for the wavenumbers or wave frequencies. The
wave frequency is a function of wavenumber $k$, the nonlocal scaling
parameter $\alpha = e_0 a$and the material properties ($E, \nu,\text{ and } \rho$ of the
nanorod. This shows a non-linear relation between wavenumber
and wave frequency i.e., the obtained axial waves in nanorod are
dispersive in nature. The present results are compared with
the results obtained from classical continuum model, second and fourth order strain gradient models, stress gradient model and Born-K\'arm\'an model. If $\alpha = 0$ and $\nu = 0$ the wavenumber is directly
proportional to wave frequency, which will give a non-dispersive
wave behaviour (explained in \cite{doyle1989wave})

\subsection*{Numerical results and discussion}
For the present analysis, a single-walled carbon nanotube is
assumed as a nanorod. The values of the radius, thickness, Young’s
modulus and density are assumed as 3.5 nm, 0.35 nm, 1.03 TPa,
and 2300 $\mathrm{kg/m^3}$ , respectively.
The present results are compared with the results obtained from
the following models:\\
\begin{itemize}
\item \textit{Classical/Local continuum model}:
	\begin{itemize}
	\item Constitutive relation:\\
		\begin{equation}
		\sigma(x) = E \varepsilon(x)
		\end{equation}
	\item Governing differential equation:\\
		\begin{equation}
			\dfrac{\partial^2 u(x,t)}{\partial x^2} = \dfrac{\rho}{E} 					\dfrac{\partial^2 u(x,t)}{\partial t^2}
		\end{equation}
	\item Dispersion relation:\\
		\begin{equation}
			k^2 + \dfrac{\rho}{E} \omega^2 = 0
		\end{equation}
	\end{itemize}
\item \textit{Nonlocal second order strain gradient model}:\\
	\begin{itemize}
	\item Constitutive equation:\\
	\begin{equation}
		\sigma(x) = E \left[ \varepsilon(x) + \alpha^2 \dfrac{\partial^2 \varepsilon(x)}		{\partial x^2} \right]
	\end{equation}
	\item Governing differential equation:\\
	\begin{equation}
	\alpha^2 \dfrac{\partial^4 u(x,t)}{\partial x^4} + \dfrac{\partial^2 u(x,t)}{\partial x^2} = \dfrac{\rho}{E} \dfrac{\partial^2 u(x,t)}{\partial t^2}
	\end{equation}
	\item Dispersion Relation:\\
	\begin{equation}
	\alpha^2 k^4 - k^2 + \frac{\rho}{E} \omega^2 = 0
	\end{equation}
	\end{itemize}
\item \textit{Nonlocal fourth order strain gradient model}:\\
	\begin{itemize}
	\item Constitutive equation:\\
	\begin{equation}
		\sigma(x) = E \left( \varepsilon(x) + \alpha^2 \dfrac{\partial^2 \varepsilon(x)}{\partial x^2}+ \alpha^4 \dfrac{\partial^4 \varepsilon(x)}{\partial x^4} \right]
	\end{equation}
	\item Governing differential equation:\\
	\begin{equation}
	\alpha^4 \dfrac{\partial^6 u(x,t)}{\partial x^6} + \alpha^2 \dfrac{\partial^4 u(x,t)}{\partial x^4} + \dfrac{\partial^2 u(x,t)}{\partial x^2} = \dfrac{\rho}{E} \dfrac{\partial^2 u(x,t)}{\partial t^2}
	\end{equation}
	\item Dispersion Relation:\\
	\begin{equation}
	-\alpha^4 k^6 + \alpha^2 k^4 - k^2 + \frac{\rho}{E} \omega^2 = 0
	\end{equation}
	\end{itemize}
\item \textit{Nonlocal stress gradient model}:\\
	\begin{itemize}
	\item Constitutive equation:\\
	\begin{equation}
		\sigma(x) - \alpha^2 \dfrac{\partial^2 \sigma(x)}{\partial x^2} = E \varepsilon(x)
	\end{equation}
	\item Governing differential equation:\\
	\begin{equation}
	\dfrac{\partial^2 u(x,t)}{\partial x^2} + \dfrac{\rho}{E} \dfrac{\partial^4 u(x,t)}{\partial x^2 \partial t^2} = \dfrac{\rho}{E} \dfrac{\partial^2 u(x,t)}{\partial t^2}
	\end{equation}
	\item Dispersion Relation:\\
	\begin{equation}
	-k^2 + \frac{\rho}{E}\alpha^2 \omega^2 k^2 +\frac{\rho}{E}\omega^2 = 0
	\end{equation}
	\end{itemize}
\item \textit{Born-K\'arm\'an model}:\\
	\begin{itemize}
	\item Dispersion Relation:\\
	\begin{equation}
	\omega = \frac{2}{a}\sqrt{\frac{E}{\rho}} sin\left(\frac{k\times a}{2}\right)
	\end{equation}
	\end{itemize}
\item \textit{Non-local stress gradient model (including lateral inertia)}:
	\begin{itemize}
	\item Fundamental governing equation:\\
	\begin{equation}
E \frac{\partial^2 u}{\partial x^2} = \rho \frac{\partial^2 u}{\partial t^2} - \rho \nu^2 \xi^2 \frac{\partial^4 u}{\partial x^2 \partial t^2} - \alpha^2 \rho \frac{\partial^4 u}{\partial x^2 \partial t^2} + \alpha^2 \rho \nu^2 \xi^2 \frac{\partial^6 u}{\partial x^4 \partial t^2}	
	\end{equation}
	\item Dispersion relation:\\
	\begin{equation}
	\omega = \sqrt{\dfrac{E}{\rho}} \dfrac{k}{\sqrt{(1+k^2 \alpha^2)(1+k^2 \xi^2 \nu^2)}}
	\end{equation}
	\end{itemize}
\end{itemize}